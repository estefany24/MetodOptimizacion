\documentclass{beamer}
\usetheme[progressbar=frametitle]{metropolis}

% Paquetes esenciales
\usepackage[spanish]{babel}
\usepackage[utf8]{inputenc}
\usepackage[T1]{fontenc}
\usepackage{xcolor}

% Definir colores personalizados
\definecolor{verdeLume}{RGB}{27,117,73}

% Colores personalizados en la presentación
\setbeamercolor{title}{fg=white,bg=verdeLume}
\setbeamercolor{frametitle}{fg=white,bg=verdeLume}
\setbeamercolor{progress bar}{fg=verdeLume}
\setbeamercolor{structure}{fg=verdeLume}

% Datos de la presentación


\begin{document}
\begin{frame}{}
\centering
\vspace{1.5cm}

% Título principal
{\Huge \textbf{Effective Written Documents}}\\[1em]



% Lista de integrantes
{\normalsize
\textbf{Group Members:} \\[0.5em]
Caceres Tacora Lizbeth Estefany \\ 
Umiña Machaca Beatriz\\
Ramirez Puma Mario Wilfredo\\
Mamani Cruz Yonhel \\

}
\end{frame}





%PARTE DE BEATRIZ 



\begin{frame}{1. ¿Por qué planificar?}
Planificar antes de escribir evita bloqueos, ahorra tiempo y mejora la calidad del documento. Cuanto menos tiempo tengamos, más importante es la planificación.
\begin{itemize}
    \item Permite claridad desde el inicio.
    \item Reduce iteraciones innecesarias.
    \item Anticipa lo que se necesita saber y decir.
\end{itemize}
\end{frame}
\begin{frame}{2. Las 5W + H: La base de la planificación}
Responder estas preguntas nos guía en el contenido, enfoque y estructura:
\begin{itemize}
    \item \textbf{Why:} ¿Cuál es el propósito?
    \item \textbf{Who:} ¿A quién va dirigido?
    \item \textbf{What:} ¿Qué información incluir?
    \item \textbf{When / Where:} ¿Cuándo y dónde se usará?
    \item \textbf{How:} Estrategia derivada de las anteriores.
\end{itemize}
\textit{Las más difíciles: Why, Who y What. Son claves para definir el contenido.}
\end{frame}
\begin{frame}{3. Definir el propósito del documento}
El propósito es el corazón del documento. No es solo una razón personal, sino una guía que da dirección y sentido.
\begin{itemize}
    \item Responde: ¿Qué debe hacer la audiencia tras leer?
    \item Permite medir el éxito de la comunicación.
    \item Evita motivaciones vagas o personales (como “impresionar”).
\end{itemize}
\textit{Un propósito claro enfocado en la audiencia facilita todas las decisiones posteriores.}
\end{frame}
\begin{frame}{4. Audiencia: ¿A quién escribo?}
Clasificar a los lectores ayuda a adaptar el contenido:
\begin{itemize}
    \item \textbf{Especialistas:} Quieren detalles técnicos.
    \item \textbf{No especialistas:} Requieren contexto y explicación.
    \item \textbf{Primarios:} Participan directamente.
    \item \textbf{Secundarios:} Lo leen después o de forma indirecta.
\end{itemize}
\textit{Usar lenguaje adecuado y ofrecer contexto permite que todos comprendan.}
\end{frame}
\begin{frame}{5. Audiencia mixta: cómo abordarla}
Escribir para diferentes tipos de lectores requiere equilibrio:
\begin{itemize}
    \item Información por capas: de general a específico.
    \item Cada oración debe aportar algo nuevo para todos.
    \item Evitar jergas; usar términos técnicos definidos.
    \item Agregar glosario solo si es necesario, y anunciarlo.
\end{itemize}
\textit{Ejemplo: “El departamento IR (Recursos de Información) fue recientemente lanzado.”}
\end{frame}
\begin{frame}{6. Selección del contenido}
No se trata de contar todo lo que hicimos, sino lo que el lector necesita saber.
\begin{enumerate}
    \item Define primero tus conclusiones clave.
    \item Selecciona los hallazgos que las respaldan.
    \item Describe solo lo relevante del trabajo.
\end{enumerate}
\textit{Evita el orden cronológico. Elige según el impacto, no el proceso.}
\end{frame}
\begin{frame}{7. Recomendaciones finales}
\begin{itemize}
    \item Planifica antes de escribir: evita errores y bloqueos.
    \item Define un propósito claro centrado en la audiencia.
    \item Adapta el contenido según tipo de lector.
    \item Elige qué incluir basándote en lo que aporta al propósito.
\end{itemize}
\textit{Un documento eficaz logra que la audiencia actúe, no solo que entienda.}
\end{frame}

%PARTE DE MARIO 
%-------------------------------------------------------------------
%-------------------------------------------------------------------
\begin{frame}
\frametitle{La Base del Diseño Efectivo de Documentos}

\begin{block}{Principio Fundamental}
\textbf{COLOCAR PRIMERO lo que los lectores están principalmente interesados}
\end{block}

\begin{columns}[T]
\begin{column}{0.48\textwidth}
\textbf{Interés de los Lectores:}
\begin{itemize}
    \item Motivación del trabajo
    \item Resultado del trabajo
    \item Contexto más amplio
    \item Relevancia personal
\end{itemize}
\end{column}

\begin{column}{0.48\textwidth}
\textbf{Enfoque de los Autores:}
\begin{itemize}
    \item Proceso cronológico
    \item Detalles del trabajo
    \item Dedicación proporcional al tiempo invertido
    \item Esfuerzo centrado en el cuerpo
\end{itemize}
\end{column}
\end{columns}

\vspace{0.5cm}
\centering{\textit{Los documentos efectivos están orientados a la audiencia, no son egocéntricos}}
\end{frame}

%--------------------------------------------------------------------------------
% DIAPOSITIVA 2
%--------------------------------------------------------------------------------
\begin{frame}
\frametitle{El Problema con la Estructura Cronológica}

\begin{columns}[T]
\begin{column}{0.48\textwidth}
\textbf{Estructura Tradicional:}
\begin{enumerate}
    \item Introducción (el antes)
    \item Cuerpo (el durante)
    \item Conclusión (el después)
\end{enumerate}

\vspace{0.3cm}
\textbf{Lo que hacen los lectores:}
\begin{itemize}
    \item Leen la introducción
    \item Saltan directamente a la conclusión
    \item Revisan selectivamente el cuerpo
\end{itemize}
\end{column}

\begin{column}{0.48\textwidth}
\textbf{Consecuencias:}
\begin{itemize}
    \item Estructura subóptima para lectores
    \item Requiere navegación innecesaria
    \item Dificulta captar la visión global
    \item No optimizada para lectores secundarios
    \item Ignora que los lectores buscan primero relevancia y resultado
\end{itemize}
\end{column}
\end{columns}

\vspace{0.3cm}
\centering{\textbf{Una estructura cronológica es sencilla para los autores pero ineficaz para los lectores}}
\end{frame}

%--------------------------------------------------------------------------------
% DIAPOSITIVA 3
%--------------------------------------------------------------------------------
\begin{frame}
\frametitle{Diseñando Documentos Cortos}

\begin{columns}
\begin{column}{0.48\textwidth}
\begin{tikzpicture}
\draw[fill=blue!10] (0,0) rectangle (4,0.8) node[pos=.5] {Encabezado};
\draw[fill=green!10] (0,-1) rectangle (4,1) node[pos=.5] {Prólogo};
\draw[fill=orange!10] (0,-2.1) rectangle (4,1) node[pos=.5] {Resumen};
\draw[fill=gray!10] (0,-3.2) rectangle (4,1) node[pos=.5] {Cuerpo};
\end{tikzpicture}
\end{column}

\begin{column}{0.48\textwidth}
\textbf{Características:}
\begin{itemize}
    \item Pocas páginas de extensión
    \item Audiencia homogénea
    \item Primera página contiene lo esencial
    \item No requiere resumen separado
    \item Equivalente a introducción + conclusión adelantada
\end{itemize}

\vspace{0.3cm}
\textbf{Ejemplos:}
\begin{itemize}
    \item Cartas de negocios
    \item Informes breves
    \item Memorandos técnicos
\end{itemize}
\end{column}
\end{columns}

\vspace{0.3cm}
\textit{La carta de una página con anexos es un ejemplo perfecto de esta estructura}
\end{frame}

%--------------------------------------------------------------------------------
% DIAPOSITIVA 4
%--------------------------------------------------------------------------------
\begin{frame}
\frametitle{Diseñando Documentos Largos}

\begin{columns}
\begin{column}{0.35\textwidth}
\begin{tikzpicture}[scale=0.8]
\draw[fill=blue!10] (0,0) rectangle (4,0.6) node[pos=.5] {Encabezado};
\draw[fill=yellow!10] (0,-0.7) rectangle (4,0.6) node[pos=.5] {Prólogo + Resumen};
\draw[fill=green!10] (0,-1.8) rectangle (4,0.8) node[pos=.5] {Introducción};
\draw[fill=gray!10] (0,-2.9) rectangle (4,1) node[pos=.5] {Tabla de contenidos};
\draw[fill=gray!30] (0,-4) rectangle (4,1) node[pos=.5] {Cuerpo};
\draw[fill=orange!10] (0,-5.1) rectangle (4,1) node[pos=.5] {Conclusión};
\draw[fill=gray!20] (0,-6.2) rectangle (4,1) node[pos=.5] {Apéndices};
\end{tikzpicture}
\end{column}

\begin{column}{0.6\textwidth}
\textbf{Características:}
\begin{itemize}
    \item Más de pocas páginas
    \item Audiencia mixta (especialistas y no especialistas)
    \item Resumen redundante en primera página
    \item Prólogo y resumen distintos de introducción y conclusión
    \item Menos técnico que el cuerpo principal
\end{itemize}

\vspace{0.3cm}
\textbf{Ventajas:}
\begin{itemize}
    \item Lectores obtienen inmediatamente lo esencial
    \item Marco para entender los detalles posteriores
    \item Herramienta efectiva de selección
    \item Ahorra tiempo a lectores secundarios
\end{itemize}
\end{column}
\end{columns}
\end{frame}

%--------------------------------------------------------------------------------
% DIAPOSITIVA 5
%--------------------------------------------------------------------------------
\begin{frame}
\frametitle{Rompiendo el Modelo Cronológico}

\begin{columns}
\begin{column}{0.48\textwidth}
\textbf{Documento Corto: Reubicar}
\begin{tikzpicture}
\node[align=right] at (0,0) {Introducción};
\node[align=left] at (2,0) {$\rightarrow$ Prólogo};
\node[align=right] at (0,-0.5) {Cuerpo};
\node[align=left] at (2,-0.5) {$\rightarrow$ Resumen};
\node[align=right] at (0,-1) {Conclusión};
\node[align=left] at (2,-1) {$\rightarrow$ Cuerpo};
\end{tikzpicture}

\vspace{0.5cm}
\textbf{Ventajas:}
\begin{itemize}
    \item Simple de implementar
    \item Ideal para audiencias homogéneas
    \item No requiere reformulación
    \item Mantiene el documento conciso
\end{itemize}
\end{column}

\begin{column}{0.48\textwidth}
\textbf{Documento Largo: Reformular}
\begin{tikzpicture}
\node[align=right] at (0,0) {Resumen};
\node[align=left] at (2,0) {$\rightarrow$ Prólogo};
\node[align=right] at (0,-0.5) {Introducción};
\node[align=left] at (2,-0.5) {$\rightarrow$ Resumen};
\node[align=right] at (0,-1) {Cuerpo};
\node[align=left] at (2,-1) {$\rightarrow$ Introducción};
\node[align=right] at (0,-1.5) {Conclusión};
\node[align=left] at (2,-1.5) {$\rightarrow$ Cuerpo};
\node[align=right] at (0,-2) {};
\node[align=left] at (2,-2) {$\rightarrow$ Conclusión};
\end{tikzpicture}

\vspace{0.1cm}
\textbf{Ventajas:}
\begin{itemize}
    \item Adecuado para documentos complejos
    \item Efectivo para audiencias mixtas
    \item Mantiene la estructura cronológica
    \item Añade redundancia útil
\end{itemize}
\end{column}
\end{columns}

\vspace{0.3cm}
\centering{\textit{Ambas estrategias colocan lo que los lectores quieren saber primero}}
\end{frame}

%--------------------------------------------------------------------------------
% DIAPOSITIVA 6
%--------------------------------------------------------------------------------
\begin{frame}
\frametitle{Componentes de un Resumen Efectivo}

\begin{columns}[T]
\begin{column}{0.48\textwidth}
\textbf{Prólogo (El antes):}
\begin{itemize}
    \item \textbf{Contexto:} Por qué la necesidad es importante
    \item \textbf{Necesidad:} Por qué algo necesitaba hacerse
    \item \textbf{Tarea:} Lo que se emprendió para abordar la necesidad
    \item \textbf{Objeto:} Lo que cubre el documento
\end{itemize}
\end{column}

\begin{column}{0.48\textwidth}
\textbf{Resumen (El después):}
\begin{itemize}
    \item \textbf{Hallazgos:} Lo que el trabajo produjo
    \item \textbf{Conclusión:} Lo que significan los hallazgos
    \item \textbf{Perspectivas:} Lo que depara el futuro
\end{itemize}
\end{column}
\end{columns}

\vspace{0.1cm}
\begin{block}{Función del Resumen}
\begin{itemize}
    \item Contar la historia de manera compacta y orientada a la audiencia
    \item Herramienta de selección para lectores potenciales
    \item Ahorrar tiempo a quienes solo necesitan lo esencial
\end{itemize}
\end{block}
\end{frame}

%--------------------------------------------------------------------------------
% DIAPOSITIVA 7
%--------------------------------------------------------------------------------
\begin{frame}
\frametitle{Pensando en Capas Concéntricas}
\begin{block}{El Valor de Este Enfoque}
El resumen responde implícitamente a las preguntas clave que los lectores tienen en cada etapa de su lectura, facilitando la comprensión y aumentando la relevancia.
\end{block}


\vspace{0.2cm}
\begin{itemize}
    \item Cada capa tiene un enfoque diferente (documento, autores, lectores, cualquiera)
    \item Cada capa contiene elementos del "antes" y del "después"
    \item Estructura lógica que responde a las necesidades de información
\end{itemize}
\end{frame}

%--------------------------------------------------------------------------------
% DIAPOSITIVA 8
%--------------------------------------------------------------------------------
\begin{frame}
\frametitle{Transmitiendo la Motivación y el Resultado}

\begin{columns}[T]
\begin{column}{0.48\textwidth}
\textbf{¿Por qué? (Motivación):}
\begin{itemize}
    \item \textbf{Contexto:} ¿Por qué ahora?\\
    Mi/nuestra o tu situación actual
    
    \item \textbf{Necesidad:} ¿Por qué tú?\\
    Por qué esto es relevante para ti
    
    \item \textbf{Tarea:} ¿Por qué yo/nosotros?\\
    Qué tengo/tenemos que ver con esto
    
    \item \textbf{Objeto:} ¿Por qué este documento?\\
    Propósito y alcance
\end{itemize}
\end{column}

\begin{column}{0.48\textwidth}
\textbf{¿Qué? (Resultado):}
\begin{itemize}
    \item \textbf{Hallazgos:} ¿Qué?\\
    Qué resultó de la tarea realizada
    
    \item \textbf{Conclusión:} ¿Entonces qué?\\
    Qué significan estos hallazgos para ti
    
    \item \textbf{Perspectivas:} ¿Qué ahora?\\
    Qué hacer a continuación
\end{itemize}
\end{column}
\end{columns}

\vspace{0.3cm}

\end{frame}

%--------------------------------------------------------------------------------
% DIAPOSITIVA 9
%--------------------------------------------------------------------------------
\begin{frame}
\frametitle{Resúmenes Subóptimos Comunes}

\begin{columns}[T]
\begin{column}{0.28\textwidth}
\textbf{Prometedor:}
\begin{tikzpicture}
\draw[fill=green!10] (0,0) rectangle (2.5,0.5) node[pos=.5] {Contexto};
\draw[fill=green!10] (0,-0.6) rectangle (2.5,0.5) node[pos=.5] {Necesidad};
\draw[fill=green!10] (0,-1.2) rectangle (2.5,0.5) node[pos=.5] {Tarea};
\draw[fill=green!10] (0,-1.8) rectangle (2.5,0.5) node[pos=.5] {Objeto};
\draw[fill=red!10] (0,-2.4) rectangle (2.5,0.5) node[pos=.5] {\textbf{X} Hallazgos};
\draw[fill=red!10] (0,-3.0) rectangle (2.5,0.5) node[pos=.5] {\textbf{X} Conclusión};
\draw[fill=red!10] (0,-3.6) rectangle (2.5,0.5) node[pos=.5] {\textbf{X} Perspectivas};
\end{tikzpicture}
\vspace{0.2cm}
\centering{\textit{Promete pero no cumple}}
\end{column}

\begin{column}{0.28\textwidth}
\textbf{Sin previo aviso:}
\begin{tikzpicture}
\draw[fill=red!10] (0,0) rectangle (2.5,0.5) node[pos=.5] {\textbf{X} Contexto};
\draw[fill=red!10] (0,-0.6) rectangle (2.5,0.5) node[pos=.5] {\textbf{X} Necesidad};
\draw[fill=red!10] (0,-1.2) rectangle (2.5,0.5) node[pos=.5] {\textbf{X} Tarea};
\draw[fill=red!10] (0,-1.8) rectangle (2.5,0.5) node[pos=.5] {\textbf{X} Objeto};
\draw[fill=green!10] (0,-2.4) rectangle (2.5,0.5) node[pos=.5] {Hallazgos};
\draw[fill=green!10] (0,-3.0) rectangle (2.5,0.5) node[pos=.5] {Conclusión};
\draw[fill=green!10] (0,-3.6) rectangle (2.5,0.5) node[pos=.5] {Perspectivas};
\end{tikzpicture}
\vspace{0.2cm}
\centering{\textit{Aparece sin contexto}}
\end{column}

\begin{column}{0.28\textwidth}
\textbf{Egocéntrico:}
\begin{tikzpicture}
\draw[fill=red!10] (0,0) rectangle (2.5,0.5) node[pos=.5] {\textbf{X} Contexto};
\draw[fill=red!10] (0,-0.6) rectangle (2.5,0.5) node[pos=.5] {\textbf{X} Necesidad};
\draw[fill=green!10] (0,-1.2) rectangle (2.5,0.5) node[pos=.5] {Tarea};
\draw[fill=red!10] (0,-1.8) rectangle (2.5,0.5) node[pos=.5] {\textbf{X} Objeto};
\draw[fill=green!10] (0,-2.4) rectangle (2.5,0.5) node[pos=.5] {Hallazgos};
\draw[fill=red!10] (0,-3.0) rectangle (2.5,0.5) node[pos=.5] {\textbf{X} Conclusión};
\draw[fill=red!10] (0,-3.6) rectangle (2.5,0.5) node[pos=.5] {\textbf{X} Perspectivas};
\end{tikzpicture}
\vspace{0.2cm}
\centering{\textit{Ignora el interés del lector}}
\end{column}
\end{columns}

\vspace{0.1cm}
\begin{block}{Problema Común}
Estos tres tipos de resúmenes hacen que los lectores se pregunten: ¿Por qué debería importarme esto? ¿Qué significa para mí? ¿Qué debo hacer al respecto?
\end{block}
\end{frame}

%--------------------------------------------------------------------------------
% DIAPOSITIVA 10
%--------------------------------------------------------------------------------
\begin{frame}
\frametitle{La Tarea y el Objeto: Diferencias Clave}

\begin{columns}[T]
\begin{column}{0.48\textwidth}
\textbf{La Tarea:}
\begin{itemize}
    \item Se centra en el trabajo realizado
    \item Sujeto: los autores (\textit{nosotros})
    \item Tiempo verbal: pasado
    \item Ejemplo: \textit{Desarrollamos un nuevo método para...}
    \item Conecta con la necesidad: \textit{Para aumentar la velocidad, rediseñamos...}
\end{itemize}

\vspace{0.1cm}
\textbf{Formulación ideal:}
\begin{itemize}
    \item Voz activa
    \item Primera persona (nosotros)
    \item Verbo preciso y signifi
\end{itemize}
\end{column}

\begin{column}{0.48\textwidth}
\textbf{El Objeto:}
\begin{itemize}
    \item Se centra en el documento
    \item Sujeto: el documento (\textit{este informe})
    \item Tiempo verbal: presente
    \item Ejemplo: \textit{Este documento presenta...}
    \item Orienta sobre contenido y estructura
\end{itemize}

\vspace{0.1cm}
\textbf{Formulación ideal:}
\begin{itemize}
    \item Documento como sujeto gramatical
    \item Tiempo presente
    \item Verbo que refleja la función comunicativa
\end{itemize}
\end{column}
\end{columns}

\vspace{0.3cm}
\begin{block}{Error Común a Evitar}
\textit{En este informe, medimos...} mezcla incorrectamente lugares y tiempos: no se mide nada en el informe sino en el laboratorio o campo.
\end{block}
\end{frame}

%--------------------------------------------------------------------------------
% DIAPOSITIVA 11
%--------------------------------------------------------------------------------
\begin{frame}
\frametitle{Consideraciones sobre el Resumen}

\begin{columns}[T]
\begin{column}{0.48\textwidth}
\textbf{¿Cuándo incluir una necesidad?}
\begin{itemize}
    \item \textbf{Siempre}, incluso si parece conocida
    \item Beneficios:
    \begin{itemize}
        \item Refresca memoria de lectores primarios
        \item Establece relación con los lectores
    \end{itemize}
\end{itemize}

\vspace{0.1cm}
\textbf{¿Por qué motivación en el resumen?}
\begin{itemize}
    \item El resumen es herramienta de selección
    \item Sin motivación, los lectores pueden abandonar
\end{itemize}
\end{column}

\begin{column}{0.48\textwidth}
\textbf{¿Resumen técnico o accesible?}
\begin{itemize}
    \item Hacer resúmenes comprensibles para todos los lectores potenciales
    \item Permitir que los lectores decidan racionalmente si el documento les servirá
\end{itemize}

\vspace{0.1cm}
\textbf{¿Escribir el resumen primero o último?}
\begin{itemize}
    \item \textbf{Primero}: mejora la estructura del documento
    \item \textbf{Último}: mejora la calidad del resumen
\end{itemize}
\end{column}
\end{columns}

\end{frame}

%--------------------------------------------------------------------------------
% DIAPOSITIVA 12
%--------------------------------------------------------------------------------
\begin{frame}
\frametitle{El Valor de los Componentes Globales}

\begin{columns}[T]
\begin{column}{0.48\textwidth}
\textbf{Beneficios para los lectores:}
\begin{itemize}
    \item Ahorran tiempo y esfuerzo
    \item Proporcionan contexto antes de los detalles
    \item Ayudan a decidir si leer el documento completo
    \item Facilitan la selección de partes relevantes
    \item Estructuran mentalmente la información posterior
\end{itemize}
\end{column}

\begin{column}{0.48\textwidth}
\textbf{Aplicaciones:}
\begin{itemize}
    \item Documentos de todo tipo y extensión
    \item Elementos del documento (capítulos, secciones)
    \item Reportes técnicos y científicos
    \item Documentos de negocios
    \item Actas de reuniones
    \item Especificaciones técnicas
    \item Procedimientos
\end{itemize}
\end{column}
\end{columns}
\end{frame}

%--------------------------------------------------------------------------------
% DIAPOSITIVA 13
%--------------------------------------------------------------------------------
\begin{frame}
\frametitle{Casos Especiales y Variaciones}

\begin{columns}[T]
\begin{column}{0.48\textwidth}
\textbf{Artículos de revisión:}
\begin{itemize}
    \item \textbf{Tarea}: Puede enfocarse en:
    \begin{itemize}
        \item El trabajo de otros investigadores
        \item El trabajo de revisión propio
    \end{itemize}
    \item \textbf{Necesidad}: Debe adaptarse 
    \item \textbf{Formulación}: \textit{Durante los últimos veinte años, los investigadores han...}
\end{itemize}

\vspace{0.2cm}
\textbf{Documentos detallados:}
\begin{itemize}
    \item Actas de reuniones
    \item Especificaciones
    \item Procedimientos
\end{itemize}
\end{column}

\begin{column}{0.48\textwidth}
\textbf{Adaptaciones para documentos detallados:}
\begin{itemize}
    \item \textbf{Prólogo}: Explica por qué el documento existe (por qué la reunión era necesaria, etc.)
    \item \textbf{Resumen}: Se centra en lo que la audiencia necesita saber principalmente:
    \begin{itemize}
        \item Decisiones y acciones (reuniones)
        \item Aspectos novedosos (especificaciones)
        \item Visión general del sistema o dispositivo
    \end{itemize}
\end{itemize}

\end{column}
\end{columns}
\end{frame}

%--------------------------------------------------------------------------------
% DIAPOSITIVA 14
%--------------------------------------------------------------------------------
\vspace{0.5cm}
\begin{block}{Principio Fundamental}
Un componente global debe tener sentido por sí solo y ser independiente, incluso si es redundante con otras partes del documento.
\end{block}

\vspace{0.3cm}
\centering{\textit{Para cada documento debemos preguntarnos: ¿Qué es lo que mis lectores necesitan saber primero y ante todo?}}
\vspace{0.2cm}
\textbf{Objetivo vs Necesidad:}
\begin{itemize}
    \item Un objetivo es solo la mitad de una necesidad
    \item Refleja situación deseada pero no la actual
    \item No justifica por sí solo una tarea
\end{itemize}

%-------------------------------------------------------------------
%MI PARTE
\begin{frame}{ .. }
\centering
\vspace{0.5cm} % espacio superior opcional
{\Huge \textbf{Redacción Efectiva de Documentos}}
\end{frame}

% CONTENIDO
\begin{frame}{Documentos cuidadosamente redactados}
\begin{itemize}
  \item ¿Cada párrafo debe transmitir un mensaje?
  \item ¿Dificultades para empezar con el mensaje principal?
  \item ¿Existe una longitud máxima para los párrafos?
\end{itemize}
\end{frame}

\begin{frame}{Redacción y desarrollo de mensajes}
\begin{itemize}
  \item Los párrafos son los componentes estructurales esenciales 
  \begin{itemize}
      \item En primer lugar, los párrafos —al igual que los documentos— cuentan su propia historia: un párrafo eficaz puede entenderse incluso fuera de contexto, mientras que muchas oraciones individuales no.
      \item En segundo lugar, los párrafos transmiten mensajes: cada uno debe expresar y desarrollar una idea específica.

  \end{itemize}
  \item Estructura interna del párrafo
  \begin{itemize}
      \item Una buena oración inicial, como un buen resumen, permite al lector saber de qué trata el párrafo y decidir si desea leerlo.
      \item Las palabras clave en esa primera oración ayudan a los lectores a localizar rápidamente los párrafos más relevantes o a volver a encontrar una información que recuerdan haber leído.

  \end{itemize}
  \item ¿Deben los párrafos ser siempre paralelos o secuenciales?
  \item ¿La estructura paralela no resulta aburrida?
\end{itemize}
\end{frame}

\begin{frame}{Listas eficaces}
\begin{itemize}
  \item Incluir pocos elementos (cinco o menos), para facilitar su lectura visual.
  \item Introducir la lista con una cláusula que indique de qué trata.
  \item Usar una estructura gramatical paralela en todos los elementos.
  \item Asegurarse de que los elementos continúan gramaticalmente la introducción.
  \item El uso de viñetas no exime del cumplimiento de las reglas gramaticales.

\end{itemize}
\end{frame}

\begin{frame}{Uso de la voz activa y el tema como sujeto}
\begin{itemize}
  \item El sujeto de la voz activa puede ser un objeto inanimado: \textit{Este estudio presenta..., Los resultados muestran...}
  \item Incluso usando “nosotros”, el verbo expresa el tema: \textit{Decidimos analizar....}

\end{itemize}
\end{frame}

\begin{frame}{Preguntas importantes}
\begin{itemize}
  \item ¿Existe una longitud máxima para las oraciones?
  \item ¿Cómo puedo corregir un sujeto demasiado largo, manteniéndolo en posición de sujeto?
  \item ¿Debo confiar en fórmulas de legibilidad?
  \item ¿La voz pasiva siempre es subóptima?
   \item ¿Puedo escribir "los autores" en lugar de "nosotros"?
   \item ¿Puedo aclarar el agente con una cita?
   \item ¿Hay usos ineficaces de la primera persona?
\end{itemize}
\end{frame}





%PARTE DE JONEL 
%-------------------------------------------------------------------
%-------------------------------------------------------------------
\begin{frame}{ .. }
\centering
\vspace{0.5cm} % espacio superior opcional
{\Huge \textbf{Formato del Documento}}
\end{frame}

\begin{frame}{1. El formato es estructura, no estética}
\begin{itemize}
    \item El formato eficaz revela la estructura del contenido sin distraer al lector.
    \item La estructura visual se basa principalmente en la \textbf{disposición espacial}, no en el aspecto visual.
    \item El uso del espacio (márgenes, interlineado, alineación) puede comunicar mucho sin necesidad de colores o estilos llamativos.
    \item La estética es secundaria; la elegancia surge de la simplicidad y coherencia.
\end{itemize}
\end{frame}

% Diapositiva 2
\begin{frame}{2. Diseño visual y uso del espacio}
\begin{itemize}
    \item La página tiene dos dimensiones: úsala para enriquecer la estructura visual.
    \item Un diseño en columnas puede facilitar la inserción de imágenes o notas al lado del texto.
    \item Diseñar una página requiere una visión global, como una \textbf{cuadrícula de diseño} que guíe posiciones y proporciones.
\end{itemize}
\end{frame}

% Diapositiva 3
\begin{frame}{3. Consistencia y jerarquía visual}
\begin{itemize}
    \item El diseño debe ser intuitivo, siguiendo principios de proximidad, similitud y secuencia visual.
    \item Formatea ítems iguales de forma idéntica y diferentes de forma claramente distinta.
    \item Indica jerarquías con tamaño, peso o espaciado.
    \item Evita el uso excesivo de énfasis: mientras más elementos se destacan, menos destacan.
\end{itemize}
\end{frame}

% Diapositiva 4
\begin{frame}{4. Tipografía y color}
\begin{itemize}
    \item Usa tipos de letra familiares y legibles; evita mezclar demasiados estilos.
    \item Elige un ancho de texto proporcional al tamaño de fuente (regla del alfabeto).
    \item Justificar el texto puede dar bloques más limpios, pero afecta la naturalidad del espaciado.
    \item Usa el color con moderación y solo cuando aporte valor estructural.
\end{itemize}
\end{frame}

% Diapositiva 5
\begin{frame}{5. Simplicidad, armonía y énfasis}
\begin{itemize}
    \item La simplicidad se logra con restricciones: un solo tipo de letra, pocos tamaños, sin efectos innecesarios.
    \item Para destacar, prefiere el \textbf{espaciado} antes que estilos como subrayado.
    \item Coordina el tamaño y posición de todos los elementos con una cuadrícula base.
    \item El formato debe \textbf{reforzar la estructura del texto}, no reemplazarla.
\end{itemize}
\end{frame}


%-------------------------------------------------------------------
%PARTE DE BEATRIZ 



\frame{\titlepage}


\section{El ciclo de la creación de documentos}
\begin{frame}
\frametitle{El ciclo de la creación de documentos}

\textbf{Conceptos clave:}
\begin{description}
\item[Planificación:] Definir el propósito, audiencia y alcance del documento
\item[Diseño:] Organizar la estructura, jerarquía y flujo de la información
\item[Redacción:] Convertir ideas en palabras siguiendo la estructura planificada
\item[Formato:] Dar presentación visual adecuada al contenido
\item[Revisión:] Proceso de mejora iterativa del contenido y la forma
\item[Documento terminado:] Producto final optimizado y listo para su propósito
\end{description}

\vspace{0.5cm}
\textbf{La revisión es parte integral del proceso, no un añadido opcional.}
\end{frame}


\section{Por qué revisar es fundamental}
\begin{frame}
\frametitle{Por qué revisar es fundamental}

\textbf{} La revisión es el proceso deliberado de evaluación y mejora que garantiza la calidad y efectividad de la comunicación.

\vspace{0.5cm}
\begin{itemize}
\item Los documentos rara vez son óptimos en su primer borrador
\item Una buena revisión puede marcar la diferencia entre un documento mediocre y uno excelente
\item Refleja la calidad de nuestro trabajo: documentos descuidados sugieren trabajo descuidado
\end{itemize}
\end{frame}


\section{La revisión como proceso iterativo}
\begin{frame}
\frametitle{La revisión como proceso iterativo}

\textbf{} La revisión no es lineal sino cíclica, pudiendo requerir reconsiderar decisiones previas.

\vspace{0.3cm}
\begin{enumerate}
\item Revisar puede significar retomar:
   \begin{itemize}
   \item \textbf{Planificación:} Repensar objetivos y audiencia
   \item \textbf{Diseño:} Reorganizar la estructura del documento
   \item \textbf{Redacción:} Mejorar la expresión de ideas
   \item \textbf{Formato:} Ajustar la presentación visual
   \end{itemize}
\vspace{0.3cm}
\item Los cambios en etapas anteriores impactan todas las posteriores
\end{enumerate}
\end{frame}


\section{Probando tu documento con lectores}
\begin{frame}
\frametitle{Probando tu documento con lectores}

\textbf{} El testing es el proceso de exponer el documento a lectores representativos para identificar áreas de mejora desde la perspectiva del receptor.

\vspace{0.5cm}
\begin{itemize}
\item \textbf{Para claridad:} Busca lectores no especializados que detecten ambigüedades
\item \textbf{Para precisión:} Consulta a especialistas que verifiquen exactitud técnica
\item \textbf{Para corrección lingüística:} Recurre a personas con dominio del idioma que identifiquen errores
\end{itemize}
\end{frame}
\section{Consejos para recibir retroalimentación}
\begin{frame}
\frametitle{Consejos para recibir retroalimentación}

\textbf{} La receptividad a la retroalimentación es la capacidad de escuchar, procesar y valorar las impresiones de los lectores sin actitud defensiva.

\vspace{0.3cm}
\begin{itemize}
\item Sé receptivo, no defensivo: la crítica es sobre el texto, no sobre ti
\item Evita justificar tu texto: si requiere justificación, probablemente necesite mejoras
\item Entiende que "esto no está claro" significa "yo lo encontré poco claro" - es una experiencia real del lector
\item Recuerda: Si un lector tuvo problemas, otros también podrían tenerlos
\end{itemize}
\end{frame}
\section{Errores comunes al revisar}
\begin{frame}
\frametitle{Errores comunes al revisar}

\textbf{} Los obstáculos sistemáticos que impiden una revisión efectiva suelen ser actitudinales más que técnicos.

\vspace{0.3cm}
\begin{itemize}
\item \textbf{Saltarse la revisión:} Considerar el primer borrador como producto final
\item \textbf{Mala gestión del tiempo:} No reservar tiempo suficiente para iterar y mejorar
\item \textbf{Elegir revisores inadecuados:} Buscar confirmación en vez de evaluación honesta
\item \textbf{Actitud defensiva:} Rechazar comentarios en lugar de considerarlos oportunidades de mejora
\item \textbf{Resistencia al cambio:} Preferir mantener lo escrito a pesar de sus deficiencias
\end{itemize}
\end{frame}

\section{Proofreading: Corrigiendo el documento}
\begin{frame}
\frametitle{Proofreading: Corrigiendo el documento}

\textbf{} El proofreading es la verificación final que garantiza la ausencia de errores superficiales que podrían distraer al lector o reducir la credibilidad.

\vspace{0.3cm}
\begin{itemize}
\item Corregir tu propio documento es difícil debido a la "ceguera de autor"
\item Técnicas efectivas:
  \begin{itemize}
  \item Distanciamiento temporal: deja pasar tiempo entre escritura y corrección
  \item Cambio de perspectiva: modifica la apariencia visual del texto
  \item Enfoque sistemático: revisa en varias pasadas, un aspecto a la vez
  \item Uso crítico de herramientas: aprovecha correctores pero evalúa sus sugerencias
  \end{itemize}
\end{itemize}
\end{frame}
\section{Mejorando sistemáticamente}
\begin{frame}
\frametitle{Mejorando sistemáticamente}

\textbf{} La mejora sistemática implica aplicar un proceso estructurado y basado en patrones para perfeccionar el documento.

\vspace{0.3cm}
\begin{itemize}
\item Inventario de errores: mantén un registro personal de errores frecuentes
\item Aprendizaje continuo: considera cada revisión como una oportunidad de desarrollo
\item Priorización estratégica: aborda primero los cambios estructurales, después los detalles
\item Perspectiva holística: utiliza una copia impresa para evaluar la coherencia global
\end{itemize}
\end{frame}

\section{Proporcionando retroalimentación útil}
\begin{frame}
\frametitle{Proporcionando retroalimentación útil}

\textbf{} La retroalimentación efectiva es específica, constructiva y orientada a soluciones, presentada de manera que facilite su aceptación.

\vspace{0.3cm}
\begin{itemize}
\item \textbf{Claridad de propósito:} Define explícitamente qué aspectos estás evaluando
\item \textbf{Sistema de codificación visual:}
  \begin{itemize}
  \item \textcolor{red}{Rojo:} errores que comprometen la efectividad del documento
  \item \textcolor{green}{Verde:} cambios que mejorarían pero no son críticos
  \item \textcolor{black}{Negro:} observaciones contextuales o explicativas
  \end{itemize}
\item \textbf{Lenguaje inclusivo:} "Podríamos mejorar esto" versus "Deberías corregir esto"
\end{itemize}
\end{frame}


\section{El proceso de revisión profesional}
\begin{frame}
\frametitle{El proceso de revisión profesional}

\textbf{} La revisión profesional es un procedimiento estructurado que maximiza la calidad del documento final mediante pasos bien definidos.

\vspace{0.3cm}
\begin{enumerate}
\item Acuerdo inicial: establecer expectativas claras sobre el alcance y profundidad de la revisión
\item Planificación temporal: determinar plazos realistas que permitan cambios significativos
\item Revisión estratificada: evaluar aspectos fundamentales antes que detalles menores
\item Evaluación balanceada: proporcionar panorama completo de fortalezas y debilidades
\item Retroalimentación constructiva: ofrecer críticas que faciliten la mejora
\item Definición de acciones: especificar claramente los siguientes pasos en el proceso
\end{enumerate}
\end{frame}

\end{document}

\begin{frame}
\centering
\Huge\textbf{¡Gracias!} \\
\vspace{0.5cm}

\end{frame}
\end{document}
